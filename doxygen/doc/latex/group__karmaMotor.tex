\section{Motor Part of the K\+A\+R\+MA Experiment}
\label{group__karmaMotor}\index{Motor Part of the K\+A\+R\+M\+A Experiment@{Motor Part of the K\+A\+R\+M\+A Experiment}}


Motor Control Module that allows the robot to push/draw the object and explore a tool.  


Motor Control Module that allows the robot to push/draw the object and explore a tool. 

\hypertarget{group__karmaToolProjection_intro_sec}{}\subsection{Description}\label{group__karmaToolProjection_intro_sec}
This module aims to control the robot hands in order to properly execute the push and the draw actions of an object within the K\+A\+R\+MA experiment to then learn the corresponding affordance. ~\newline
 It also enable the tool exploration.\hypertarget{group__karmaToolProjection_lib_sec}{}\subsection{Libraries}\label{group__karmaToolProjection_lib_sec}

\begin{DoxyItemize}
\item Y\+A\+RP libraries.
\end{DoxyItemize}\hypertarget{group__karmaToolProjection_parameters_sec}{}\subsection{Parameters}\label{group__karmaToolProjection_parameters_sec}
--robot {\itshape robot} 
\begin{DoxyItemize}
\item Select the robot to connect to.
\end{DoxyItemize}

--name {\itshape name} 
\begin{DoxyItemize}
\item Select the stem-\/name of the module used to open up ports. By default {\itshape name} is {\itshape karma\+Motor}.
\end{DoxyItemize}

--elbow\+\_\+set {\itshape ($<$height$>$ $<$weight$>$)}
\begin{DoxyItemize}
\item To specify how to weigh the task to keep the elbow high.
\end{DoxyItemize}\hypertarget{group__karmaToolFinder_portsa_sec}{}\subsection{Ports Accessed}\label{group__karmaToolFinder_portsa_sec}
Assume that i\+Cub\+Interface (with I\+Cartesian\+Control interface implemented) is running.\hypertarget{group__karmaToolProjection_portsc_sec}{}\subsection{Ports Created}\label{group__karmaToolProjection_portsc_sec}

\begin{DoxyItemize}
\item {\itshape /karma\+Motor/rpc} receives the information to execute the motor action as a Bottle. It manages the following commands\+:
\begin{DoxyEnumerate}
\item {\bfseries Push}\+: {\itshape \mbox{[}push\mbox{]} cx cy cz theta radius}. ~\newline
 The coordinates {\itshape (cx,cy,cz)} represent in meters the position of the object\textquotesingle{}s centroid to be pushed; {\itshape theta}, given in degrees, and {\itshape radius}, specified in meters, account for the point from which push the object, that is located onto the circle centered in {\itshape (cx,cy,cz)} and contained in the x-\/y plane. ~\newline
 The reply {\itshape \mbox{[}ack\mbox{]}} is returned as soon as the push is accomplished.
\item {\bfseries Draw}\+: {\itshape \mbox{[}draw\mbox{]} cx cy cz theta radius dist}. ~\newline
 The coordinates {\itshape (cx,cy,cz)} represent in meters the position of the object\textquotesingle{}s centroid to be drawn closer; {\itshape theta}, given in degrees, and {\itshape radius}, specified in meters, account for the point from which draw the object, that is located onto the circle centered in {\itshape (cx,cy,cz)} and contained in the x-\/y plane. The parameter {\itshape dist} specifies the length in meters of the draw action. ~\newline
 The reply {\itshape \mbox{[}ack\mbox{]}} is returned as soon as the draw is accomplished.
\item {\bfseries Virtual draw}\+: {\itshape \mbox{[}vdraw\mbox{]} cx cy cz theta radius dist}. ~\newline
 Simulate the draw without performing any movement in order to test the quality of the action. ~\newline
 The reply {\itshape \mbox{[}ack\mbox{]} val} is returned at the end of the simulation, where {\itshape val} accounts for the quality of the action\+: the lower it is the better the action is.
\item {\bfseries Tool-\/attach}\+: {\itshape \mbox{[}tool\mbox{]} \mbox{[}attach\mbox{]} arm x y z}. ~\newline
 Attach a tool to the given arm whose dimensions are specified in the frame attached to the hand. The subsequent action will make use of this tool.
\item {\bfseries Tool-\/get}\+: {\itshape \mbox{[}tool\mbox{]} \mbox{[}get\mbox{]}}. ~\newline
 Retrieve tool information as {\itshape \mbox{[}ack\mbox{]} arm x y z}.
\item {\bfseries Tool-\/remove}\+: {\itshape \mbox{[}tool\mbox{]} \mbox{[}remove\mbox{]}}. ~\newline
 Remove the attached tool.
\item {\bfseries Find}\+: {\itshape \mbox{[}find\mbox{]} arm eye}. ~\newline
 An exploration is performed which aims at finding the tool dimension. It is possible to select the arm for executing the movement as well as the eye from which the motion is observed. The reply {\itshape \mbox{[}ack\mbox{]} x y z} returns the tool\textquotesingle{}s dimensions with respect to reference frame attached to the robot hand.
\end{DoxyEnumerate}
\item {\itshape /karma\+Motor/stop}\+:i receives request for immediate stop of any ongoing processing.
\item {\itshape /karma\+Motor/vision}\+:i receives the information about the pixel corresponding to the tool tip during the tool exploration phase.
\item {\itshape /karma\+Motor/finder}\+:rpc communicates with the module in charge of solving for the tool\textquotesingle{}s dimensions.
\end{DoxyItemize}\hypertarget{group__karmaToolProjection_tested_os_sec}{}\subsection{Tested OS}\label{group__karmaToolProjection_tested_os_sec}
Windows, Linux

\begin{DoxyAuthor}{Author}
Ugo Pattacini 
\end{DoxyAuthor}
